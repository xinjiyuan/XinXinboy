\documentclass[a4paper]{article}
\usepackage{ctex}
\title{避障调研}
\date{}
\author{辛文钧}


\begin{document}
	\maketitle
	\section{避障是什么}
	避障是指移动机器人在行走过程中,通过传感器感知到在其规划路线上存在静态或动态障碍物时,按照一定的算法实时更新路径,绕过障碍物,最后到达目标点。
	
	个人理解,避障和整体的路径规划不同,它更要求实时性和准确性,它一般发生在整体路径规划的局部部分。
	
	\section{避障种类}
	现在避障种类可以分为
	\begin{enumerate}
		\item 根据场景模型的避障策略
		\item 根据案例学习的避障策略
		\item 根据行为模式的避障策略
	\end{enumerate}
	从移动机器人躲避障碍物的范围,又可分为
	\begin{enumerate}
		\item 全局避障
		\item 局部避障
	\end{enumerate}
	根据工作环境是否改变,又可分为
	\begin{enumerate}
		\item 静态避障
		\item 动态避障
	\end{enumerate}
	
	\section{避障技术种类}
	现在主要的避障技术为以下几类:
	\begin{description}
		\item[超声波避障] 通过压电或静电变送器产生一个几十kHz的超声波,
		测量超声波发射到遇到物体返回的时间测量与障碍物的距离。
		\item[红外避障] 通常使用三角测距原理。红外发射器按照一定角度发射
		红外光束,遇到物体后,光会反射,检测到反射的光,通过结构上的集合三
		角关系,可计算距离物体的距离。
		\item[激光避障] 与超声波类似。通过测量激光的飞行时间进行测距。
		\item[视觉避障] 分类很多,包括视觉图像、基于TOF的深度相机、基于
		结构光的深度相机。但深度相机需要物体主动发光,受现实太阳光等条件影
		响太大,所以大部分使用的视觉方案是基于双目视觉或单目视觉。
	\end{description}
	
	\section{视觉避障主要问题和方法}
	主要问题分为障碍物识别问题和避障策略问题。
	\subsection{障碍物识别问题}
	视觉中障碍物识别与地形环境中的障碍物分布与形状密切相关,如何实时准确的对地形环境中的障碍物进行准确识别,是能否顺利通过复杂地形的关键。
	主要的障碍物识别方法有
	\begin{itemize}
		\item 基于支持向量机的障碍物识别方法
		\item 基于神经网络的障碍物识别方法
		\item 基于PCA算法的障碍物识别方法
		\item 基于深度学习的障碍物识别方法
	\end{itemize}
	
	\subsection{避障策略}
	按照基本原理和出现顺序,可分为传统避障算法和智能避障算法。
	
	\subsubsection{传统避障算法}
	\begin{description}
		\item[Bug算法] 发现障碍物后,围着检测到的障碍物轮廓走,达到绕开目的。但算法效率较低,且有时没有考虑机器人动力学限制。
		\item[势场法(PFM)] 目标点产生引力,障碍物产生斥力,引导机器人避开障碍物。但容易出现局部极小点问题,且实时性不太理想。
		\item[向量直方图法(VFH)] 对移动机器人当前周边环境创建了一个基于极坐标表示的局部地图(形成一个直方图),实际应用的过程中会根据这个直方图首先辨识出允许机器人通过的足够大的所有空隙,然后对所有这些空隙计算其代价函数,最终选择具有最低代价函数的通路通过。
		\item[基于模板匹配的机器人避障算法] 利用之前避障产生的信息数据建立数据库,当机器人遇到避障问题时,与已有的避障模板进行拼配,进行避障。但一般适用于静态环境的避障中。
	\end{description}
		
	\subsubsection{智能避障算法}
	\begin{description}
		\item[基于神经网络的机器人避障算法] 将优化目标函数定义为轨迹点集的碰撞能量函数与距离函数的和. 通过迭代求解优化目标函数极值,使路径能避障同时趋向于最短。但有时环境难以用数学公式描述,且权值的训练不理想,会导致算法效果不理想。
		\item[基于蚁群算法的机器人避障算法] 通过仿真迭代,模拟蚁群觅食行为,得到避障路径。但存在容易产生局部最优和计算量大的缺点。
		\item[基于强化学习的机器人避障算法] 基于强化学习方法,训练得到避障路径。
		\item[基于遗传算法的机器人避障算法] 使用遗传算法,计算最优避障路径。
	\end{description}
	
\end{document}