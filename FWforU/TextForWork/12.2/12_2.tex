\documentclass[a4paper]{article}
\usepackage{ctex}
\title{上周工作总结}
\author{辛文钧}
\begin{document}
	\maketitle
	\begin{itemize}
		\item 完成第一次ros的C++程序编写,测试工程可行性。
		\item 初步了解ros的文件结构等信息,为接下来的工作做准备
		\item 明确接下来具体工作的方向(在下面进行详细解释)
	\end{itemize}
	\section{工作思考}
	我的研究是《基于视觉的机器人避障系统设计与实现》,主要研究方向为机器人的避障,
	其中需要的技术点大体如下:
	\begin{itemize}
		\item[1] 机器人根据视觉建图
		\item[2] 机器人根据识别障碍物进行避障,或路径规划
		\item[3] 在最后实现过程中,要使用强化学习的算法进行机器人行走
	\end{itemize}
	
	在现有条件中,实现的硬件条件是实验室中的turtlebot2机器人,平台条件是ros机器人
	系统,代码编写以C++为主。
	
	在思考中,想象显示环境中很多情况下机器人是在未知环境下进行避障,认为有以下问题:
	\begin{itemize}
		\item[1] 我们需要一个基于视觉的快速识别障碍物或是建图的算法
		\item[2] 我们需要一个平滑避开障碍物,继续前进的算法
	\end{itemize}
	
	其中,根据我的工作目标性,我的工作重心要放在第二个问题,至于第一个问题,由于
	turtlebot2中有自带的建图算法(gmapping算法),可暂时使用此算法代替,至于如果
	最后有兴趣改进建图部分的工作,可作为辅助研究,在空余时间研究。
\end{document}